\documentclass{article}
\usepackage[utf8]{inputenc}

\usepackage{amsthm}
\usepackage{amsmath} 
\usepackage{bbm}
\usepackage{amsfonts}
\usepackage{graphicx,color}
\usepackage[ruled, linesnumbered]{algorithm2e}
\usepackage{algorithmic}
% \usepackage[dvipsnames]{xcolor}


\usepackage{thm-restate}
\usepackage[colorlinks,linkcolor=blue,filecolor=blue,citecolor=blue,urlcolor=blue,pdfstartview=FitH,pagebackref]{hyperref}
\usepackage[nameinlink]{cleveref}


% \theoremstyle{plain}
\newtheorem{theorem}{Theorem}%[section]
%\newtheorem{corollary}[theorem]{Corollary}
\newtheorem{lem}[theorem]{Lemma}
\crefname{lem}{Lemma}{Lemmas}
\newtheorem{claim}[theorem]{Claim}
\newtheorem{observation}[theorem]{Observation}
%\newtheorem{definition}[theorem]{Definition}
%\newtheorem{conjecture}[theorem]{Conjecture}
%\newtheorem{proposition}[theorem]{Proposition}
\newtheorem{problem}{Problem}
\newtheorem{fact}[theorem]{Fact}
%\theoremstyle{remark}
%\newtheorem*{note}{Note}


% \newcommand{\Pr}{\mathbb{Pr}}
\newcommand{\eps}{\varepsilon}
\newcommand{\R}{\mathbb{R}}
\newcommand{\N}{\mathbb{N}}
\newcommand{\Z}{\mathbb{Z}}
\newcommand{\C}{\mathcal{C}}
\newcommand{\norm}[1]{\left\| #1 \right\|}
\newcommand{\NN}{\mathcal{NN}}
\newcommand{\FF}{\mathcal{F}}
\newcommand{\LL}{\mathcal{L}}
\newcommand{\DT}{\mathcal{DT}}

\definecolor{purple}{RGB}{150, 1, 255}
\definecolor{DarkGreen}{RGB}{70, 130, 30}

\newcommand{\blue}{{\color{blue}blue}}
\newcommand{\red}{{\color{red}red}}
\newcommand{\purple}{{\color{purple}purple}}
\newcommand{\comment}[1]{\CommentSty{{\color{DarkGreen}#1}}}


\title{Fault tolerant spanners}

\begin{document}
	
	
	\maketitle
	
	\section{Introduction}
	
	Let $\FF$ be a family of regions in the plane, which we call the fault regions. For a fault region $F\in \FF$ and a geometric graph $G$ on a point set $P$ , we define $G\ominus F$ to be the part of $G$ that remains after the points from $P$ inside $F$ and all edges that intersect $F$ have been removed from the graph. For simplicity, we assume that a region fault F does not contain its boundary, i.e., only vertices and edges intersecting the interior of $F$ will be affected.
	
	Let $\LL$ be a family of regions in the plane, which we call the local regions. For a fault region $L\in \LL$ and a geometric graph $G$ on a point set $P$ , we define $G\mid_F$ to be the part of $G$ contained in the interior of $F$, meaning only the vertices and edges that are fully containd in the interior of $F$.
	
	\paragraph{The problem:} Given a set $P$ of points in $\R^2$, and a family $\FF$ of regions, compute a graph $G$ that is a $t$-spanner of $P$ under any fault $F\in\FF$.
	
	\section{Complement of disk faults / disk local spanners}
	
	Let $\LL$ be the set of disks, we can use the algorithm of Abam et al. \cite{bibid}, and replace the set of edges between every pair $(A_i,B_i)$ in the SSPD with the edges of the Delaunay triangulation $\DT(P)$ with one end in $A_i$ and the other in $B_i$. We only need to prove that for any disk $d\in \LL$, and for any set $P'$ of points we have that $\DT(P')\mid_d$ is connected, as that will imply that for every pair $(A_i,B_i)$ of the SSPD, if $A_i\cap d$ and $B_i\cap d$ are not empty, there exists an edge between them in the constructed graph $G$.
	
	\begin{claim}
		For a set of points $P\subseteq \R^2$ and for any disk $d$, $\DT(P)\mid_d$ is connected.
	\end{claim}

	\begin{proof}
		We prove a different claim that immediately implies the desired one. Let $d$ be a disk with two points $p,q\in P$ on its boundary. Then there is a path between $p$ and $q$ in $\DT(P)\mid_d$.
		
		We prove by induction over the number points in the interior of $d$.
		
		$|d\cap (P\setminus \partial d)| = 0$: Then by construction of the Delaunay triangulation the edge $\{p,q\}$ is in $\DT(P)$ and is contained in the interior of $D$.\\
		
		$|d\cap (P\setminus \partial d)| > 0$: Let $x\in P$ be a point in the interior of $d$. We move the center of $d$ in the direction of $p$, shrinking $d$ in the process, until we get a disk $d'\subseteq d$ such that $x$ is on the boundary of $d'$. By induction there is a path between $p$ and $x$ in $\DT\mid_{d'}$, and since $\DT\mid_{d'}\subseteq \DT\mid_{d}$ we have that the same path exists in $\DT\mid_{d}$. The same proof gives us a path between $x$ and $q$ and thus we are done.
		
	\end{proof}

	\section{Complement of square faults / square local spanners}
	
	For the same reasons, if we can prove that for a square $s$ we have that $\DT_\infty\mid_s$, where $\DT_\infty$ is the $L_{\infty}$ norm Delaunay triangulation, is connected, we would be able to use the same algorithm almost verbatim. Fortunately, the same proof works just as well when replacing disks with squares.
	
	
	\section{Complement of convex symmetric polygonal faults / convex symmetric polygonal local spanners}
	
	Since the Delanay triangulation is well defined for the \emph{polygonal convex distance} (L.P. Chew and R.L. Drysdale. Voronoi diagrams based on convex distance functions.InProc. 1st Sympos. Comput. Geom., pages 234–244, ACM Press, 1985), we can use the exact same proof (again replacing only the distance function and the disk) to get an algorithm for this entire family of problems.
	
	\section{$\eps$-shadow}
	
	In this section we consider a weaker form of fault tolerance. Given a family $\LL$ of shapes, we say that $G$ is a $(\LL,\eps)$-local spanner if for any $L\in\LL$ we have that $G\mid_{L_{\alpha}}$, where $L_{\alpha}$ is $L$ rescaled by $\alpha$, is a $t$-spanner of $P$. For the sake of clarity we will use $\alpha = 1-\eps$.
	
	\subsection{$\beta$-fat rectangles}
	Let $\LL$ be the set of $\beta$-fat axis parallel rectangles. We repeatedly preform the algorithm for convex symmetric polygonal local spanners, where in the $i$-th iteration we use a rectangle with aspect ratio $\left(1+\eps\right)^i$, where $i\in\{0,...,\log_{1+\eps}(\beta)\}$. 
	
	Let $r$ be a $\beta$-fat rectangle, and let $(A_i,B_i)$ be a pair in an SSPD such that $A_i\cap r\neq \emptyset$, and $B_i\cap r\neq \emptyset$. We assume w.l.o.g that the height of $r$ is 1, and its width is $\beta'\in [1,\beta]$.

	Let $i\in \{0,...,\log_{1+\eps}(\beta)\}$ be an index for which $\beta' \leq (1+\eps)^i \leq \frac{\beta}{1-\eps}$ if such an index exists, then let $r'$ be the rectangle with width $\beta'$ and aspect ratio $(1+\eps)^i$, whose horizontal bisector coincides with that of $r$. Since $(1+\eps)^i \leq \frac{\beta}{1-\eps}$, we have that $r\setminus r'$ is contained within the shadow of $r$, and therefore if $r'$ contains points of both $A_i$ and $B_i$, then due to the correctness of the convex symmetric polygonal fault spanner, we will have an edge between a point in $A_I\cap r'$ and $B_i\cap r'$.
	
	We are left with proving that there exists an index $i\in \{0,...,\log_{1+\eps}(\beta)\}$ for which $\beta' \leq (1+\eps)^i \leq \frac{\beta}{1-\eps}$.
	
	$$\beta' \leq (1+\eps)^i \leq \frac{\beta}{1-\eps}$$
	
	$$\log_{1+\eps}(\beta') \leq i \leq \log_{1+\eps}\left(\frac{\beta}{1-\eps}\right)$$
	
	$$\log_{1+\eps}(\beta') \leq i \leq \log_{1+\eps}(\beta) - \log_{1+\eps}(1-\eps)$$
	
	If $\log_{1+\eps}(1-\eps)<-1$, then there must be an integer $i$ with the required properties. We now notice that $(1+\eps)^{-1}=\frac{1}{1+\eps}>(1-\eps)$ [since $1>(1-\eps)(1+\eps)=(1-\eps^2)$], and so $i$ exists.
	
	The size of the spanner is $\log_{1+\eps}(\beta)$ times the number of edges in a convex symmetric polygonal local spanner, and since $\log_{1+\eps}(\beta)=O\frac{\log(\beta)}{\eps}$, we have a spanner of size ???
	
	
	\subsection{Arbitrary rectangles}
	
	We first describe a subroutine for connecting two sets of points, $A$ and $B$, where $A$ is contained in the negative quadrant of the plane (i.e., have a negative value $x$-coordinate and a negative value $y$-coordinate), and $B$ is contained in the positive quadrant of the plane. Denote $A=\{a_i~|~ 1\leq i \leq |A|\}$, and Let
	
	$$A' = \{a_i\in A~|~ \text{the rectangle defined by the corners }(x_i,y_i)\text{ and } (0,0)\text{ contains no points of }A\}$$
	
	For every point $a_i\in A'$ we define $a_i^+=(|x_i|,|y_i|)$, and denote the rectangle defined by the corners $a_i^+$ and $(0,0)$ by $r_i$. The algorithm will connect $a_i$ to $O\left(\frac{1}{\eps^2}\right)$ points in the positive quadrant, such that each 
	
\end{document}
