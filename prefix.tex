\ifx\SoCG\undefined%
\documentclass[11pt]{article}%
\providecommand{\SoCGVer}[1]{}%
\providecommand{\NotSoCGVer}[1]{#1}%
\else%
%%%% Search the lipics/ subdirectory for style/files...
\makeatletter
\def\input@path{{lipics/}{../lipics/}}
\makeatother
%\documentclass[a4paper,USenglish,cleveref,autoref,thm-restate, nolineno ]{socg-lipics-v2021}
\documentclass[a4paper,USenglish,cleveref,autoref,thm-restate]%
{socg-lipics-v2021}

\hideLIPIcs 
\providecommand{\SoCGVer}[1]{#1}%
\providecommand{\NotSoCGVer}[1]{}%
\fi


%%%%%%%%%%%%%%%%%%%%%%%%%%%%%%%%%%%%%%%%%%%%%%%%%%%%%%%%%%%%%%%%%%   
% Conditional compilation depending on whether this is my computer or
% not.
\IfFileExists{sariel_computer.sty}{\def\sarielComp{1}}{}

\ifx\sarielComp\undefined%
\newcommand{\SarielComp}[1]{}
\newcommand{\NotSarielComp}[1]{#1}%
\else
\newcommand{\SarielComp}[1]{#1}%
\newcommand{\NotSarielComp}[1]{}%
\fi
\newcommand{\IfPrinterVer}[2]{#2}%


%%%%%%%%%%%%%%%%%%%%%%%%%%%%%%%%%%%%%%%%%%%%%%%%%%%%%%%%%%%%%%%%%% 

\NotSoCGVer{%
\usepackage[cm]{fullpage}%
}

\usepackage{amsmath}%
\usepackage{amssymb}%
\usepackage[cmyk]{xcolor}%

\NotSoCGVer{%
   \usepackage[utf8x]{inputenc}%
   \usepackage{euscript}%
}
\NotSoCGVer{%
   \usepackage[amsmath,thmmarks]{ntheorem}%
   \theoremseparator{.}%
}

%%%% Number paragraphs
%\setcounter{secnumdepth}{5}

\NotSoCGVer{%
   \usepackage{titlesec}%
   \titlelabel{\thetitle. }%
   
   \titleformat{\paragraph}[runin]
   {\normalfont\bfseries}
   {\theparagraph}
   {1em}
   {\addperiod}
   
   \newcommand{\addperiod}[1]{#1.}%
}  
%\let\oldparagraph=\paragraph
%\renewcommand\paragraph[1]{\oldparagraph{#1.}}

%#\let\oldparagraphx=\paragraph
%#\renewcommand\paragraph*[1]{\oldparagraphx{#1.}}

\usepackage{graphicx}%
\usepackage{xcolor}%
\usepackage{mleftright}%
\usepackage{xspace}%
\usepackage{hyperref}%
%\usepackage{marvosym}%
\usepackage{bm}

\usepackage{caption}%

\SarielComp{\usepackage{sariel_colors}}%

\newcommand{\hrefb}[3][black]{\href{#2}{\color{#1}{#3}}}%

\IfPrinterVer{%
   \usepackage{hyperref}%
}{%
   \usepackage{hyperref}%
   \hypersetup{%
      breaklinks,%
      ocgcolorlinks, colorlinks=true,%
      urlcolor=[rgb]{0.25,0.0,0.0},%
      linkcolor=[rgb]{0.5,0.0,0.0},%
      citecolor=[rgb]{0,0.2,0.445},%
      filecolor=[rgb]{0,0,0.4},
      anchorcolor=[rgb]={0.0,0.1,0.2}%
   }
   % \usepackage{cleveref}
}


%%%%%%%%%%%%%%%%%%%%%%%%%%%%%%%%%%%%%%%%%%%%%%%%%%%%%%%%%%%%%%%%%%%
% Biblatex....
%
\providecommand{\BibLatexMode}[1]{}
\providecommand{\BibTexMode}[1]{#1}

\ifx\UseBibLatex\undefined%
  \renewcommand{\BibLatexMode}[1]{}
  \renewcommand{\BibTexMode}[1]{#1}
\else
  \renewcommand{\BibLatexMode}[1]{#1}
  \renewcommand{\BibTexMode}[1]{}
\fi

%\usepackage{proofapnd}

% Bib latex stuff
\BibLatexMode{%
%   \usepackage[bibencoding=utf8,style=alphabetic,%
%   backend=biber,sortlocale=en_US]{biblatex}%
   \usepackage{styles/sariel_biblatex}%
   \usepackage[english]{babel}%
   \usepackage{csquotes}
   \renewcommand*{\multicitedelim}{\addcomma\space}%
}

%%%%%%%%%%%%%%%%%%%%%%%%%%%%%%%%%%%%%%%%%%%%%%%%%%%%%%%%%%%%%%%%%%%%%%%%
%%%%%%%%%%%%%%%%%%%%%%%%%%%%%%%%%%%%%%%%%%%%%%%%%%%%%%%%%%%%%%%%%%%%%%%%
% SoCG Style: Defining theorem like environments
%%%%%%%%%%%%%%%%%%%%%%%%%%%%%%%%%%%%%%%%%%%%%%%%%%%%%%%%%%%%%%%%%%%%%%%%
%%%%%%%%%%%%%%%%%%%%%%%%%%%%%%%%%%%%%%%%%%%%%%%%%%%%%%%%%%%%%%%%%%%%%%%%

\SoCGVer{%
   \theoremstyle{plain}%
   \newtheorem{FakeCounter}{FakeCounter}%[subsection]%

%   \newtheorem{conjecture}[theorem]{Conjecture}
   \newtheorem{fact}[theorem]{Fact}
%   \newtheorem{observation}[theorem]{Observation}
   \newtheorem{invariant}[theorem]{Invariant}
   \newtheorem{question}[theorem]{Question}
%   \newtheorem{proposition}[theorem]{Proposition}
   \newtheorem{prop}[theorem]{Proposition}
   \newtheorem{openproblem}[theorem]{Open Problem}

   \theoremstyle{plain}%
   \newtheorem{defn}[theorem]{Definition}
%   \newtheorem{exercise}[theorem]{Exercise}
   \newtheorem{problem}[theorem]{Problem}
   \newtheorem{xca}[theorem]{Exercise}
   \newtheorem{exercise_h}[theorem]{Exercise}
   \newtheorem{assumption}[theorem]{Assumption}%

   \newtheorem{proofof}{Proof of\!}%
}%
% SoCG style theorem block END
%%%%%%%%%%%%%%%%%%%%%%%%%%%%%%%%%%%%%%%%%%%%%%%%%%%%%%%%%%%%%%%%%%%%


% ----------------------------------------------------------------------
% ----------------------------------------------------------------------
% Defining theorem like environments
% ----------------------------------------------------------------------
% ----------------------------------------------------------------------
\NotSoCGVer{%
\theoremseparator{.}%

\theoremstyle{plain}%
\newtheorem{theorem}{Theorem}[section]

\newtheorem{lemma}[theorem]{Lemma}
\newtheorem{conjecture}[theorem]{Conjecture}
\newtheorem{corollary}[theorem]{Corollary}
\newtheorem{claim}[theorem]{Claim}%
\newtheorem{fact}[theorem]{Fact}
\newtheorem{observation}[theorem]{Observation}
\newtheorem{invariant}[theorem]{Invariant}
\newtheorem{question}[theorem]{Question}
\newtheorem{proposition}[theorem]{Proposition}
\newtheorem{prop}[theorem]{Proposition}
\newtheorem{openproblem}[theorem]{Open Problem}

\theoremstyle{plain}%
\theoremheaderfont{\sf} \theorembodyfont{\upshape}%
\newtheorem*{remark:unnumbered}[theorem]{Remark}%
\newtheorem*{remarks}[theorem]{Remarks}%
\newtheorem{remark}[theorem]{Remark}%
\newtheorem{definition}[theorem]{Definition}
\newtheorem{defn}[theorem]{Definition}
\newtheorem{example}[theorem]{Example}
\newtheorem{exercise}[theorem]{Exercise}
\newtheorem{problem}[theorem]{Problem}
\newtheorem{xca}[theorem]{Exercise}
\newtheorem{exercise_h}[theorem]{Exercise}
\newtheorem{assumption}[theorem]{Assumption}%

% Proof environment
\newcommand{\myqedsymbol}{\rule{2mm}{2mm}}

\theoremheaderfont{\em}%
\theorembodyfont{\upshape}%
\theoremstyle{nonumberplain}%
\theoremseparator{}%
\theoremsymbol{\myqedsymbol}%
\newtheorem{proof}{Proof:}%

\newtheorem{proofof}{Proof of\!}%
}
% theorem block end
%%%%%%%%%%%%%%%%%%%%%%%%%%%%%%%%%%%%%%%%%%%%%%%%%%%%%%%%%%%%%%%%%%%%


%%%%%%%%%%%%%%%%%%%%%%%%%%%%%%%%%%%%%%%%%%%%%%%%%%%%%%%%%%%%%%%%%% 5
% Color emph
\providecommand{\emphind}[1]{\emph{#1}\index{#1}}
\definecolor{nalmostblack}{rgb}{0, 0, 0.7}
\providecommand{\emphic}[2]{%
   \textcolor{nalmostblack}{%
      \textbf{\emph{#1}}}%
   \index{#2}}


\providecommand{\emphi}[1]{\emphic{#1}{#1}}

\definecolor{almostblack}{rgb}{0, 0, 0.5}
\providecommand{\emphw}[1]{{\emph{{\textcolor{almostblack}{#1}}}}}%

\providecommand{\emphOnly}[1]{\emph{\textcolor{almostblack}{\textbf{#1}}}}
% Color emph - end 
%%%%%%%%%%%%%%%%%%%%%%%%%%%%%%%%%%%%%%%%%%%%%%%%%%%%%%%%%%%%%%%%%% 5


\numberwithin{figure}{section}%
\numberwithin{table}{section}%
\numberwithin{equation}{section}%


%%%%%%%%%%%%%%%%%%%%%%%%%%%%%%%%%%%%%%%%%%%%%%%%%%%%%%%%%%%%%%%%%%%
% Sariel's thanks
%%%%%%%%%%%%%%%%%%%%%%%%%%%%%%%%%%%%%%%%%%%%%%%%%%%%%%%%%%%%%%%%%%% 

\providecommand{\tildegen}{{\protect\raisebox{-0.1cm}
      {\symbol{'176}\hspace{-0.01cm}}}}
\newcommand{\atgen}{\symbol{'100}}
\newcommand{\SarielThanks}[1]{\thanks{Department of Computer Science;
      University of Illinois; 201 N. Goodwin Avenue; Urbana, IL,
      61801, USA; {\tt sariel\atgen{}illinois.edu}; {\tt
         \url{http://sarielhp.org/}.} #1}}


%%%%%%%%%%%%%%%%%%%%%%%%%%%%%%%%%%%%%%%%%%%%%%%%%%%%%%%%%%%%%%%%%%%%%%
%    Handling references
%%%%%%%%%%%%%%%%%%%%%%%%%%%%%%%%%%%%%%%%%%%%%%%%%%%%%%%%%%%%%%%%%%%%%%

\newcommand{\HLink}[2]{\hyperref[#2]{#1~\ref*{#2}}}
\newcommand{\HLinkSuffix}[3]{\hyperref[#2]{#1\ref*{#2}{#3}}}

\newcommand{\tablab}[1]{\label{table:#1}}
\newcommand{\tabref}[1]{\HLink{Table}{table:#1}}

\newcommand{\figlab}[1]{\label{fig:#1}}
\newcommand{\figref}[1]{\HLink{Figure}{fig:#1}}

\newcommand{\thmlab}[1]{{\label{theo:#1}}}
\newcommand{\thmref}[1]{\HLink{Theorem}{theo:#1}}

\newcommand{\corlab}[1]{\label{cor:#1}}
\newcommand{\corref}[1]{\HLink{Corollary}{cor:#1}}%

\providecommand{\deflab}[1]{\label{def:#1}}
\newcommand{\defref}[1]{\HLink{Definition}{def:#1}}


\newcommand{\clmlab}[1]{\label{claim:#1}}
\newcommand{\clmref}[1]{\HLink{Claim}{claim:#1}}

%\newcommand{\apndlab}[1]{\label{apnd:#1}}
%\newcommand{\apndref}[1]{\HLink{Appendix}{apnd:#1}}

\newcommand{\seclab}[1]{\label{sec:#1}}
\newcommand{\secref}[1]{\HLink{Section}{sec:#1}}
\newcommand{\rectA}{\Mh{B}}%
\newcommand{\rectB}{\Mh{D}}%
\newcommand{\clientsY}[2]{\Mh{\mathsf{C}}\pth{#1,#2}}

\newcommand{\DW}{\times}
%\newcommand{\ConeSet}{\Mh{\mathcal{C}}}%
\newcommand{\shrinkDY}[2]{#1_{\boxminus #2}}
\newcommand{\Rects}{\Mh{\mathcal{R}}}%


\newcommand{\itemlab}[1]{\label{item:#1}}
\newcommand{\itemref}[1]{\HLinkSuffix{}{item:#1}{}}

\newcommand{\apndlab}[1]{\label{apnd:#1}}
\newcommand{\apndref}[1]{\HLink{Appendix}{apnd:#1}}

\newcommand{\remlab}[1]{\label{rem:#1}}
\newcommand{\remref}[1]{\HLink{Remark}{rem:#1}}%

\newcommand{\lemlab}[1]{\label{lemma:#1}}
\newcommand{\lemref}[1]{\HLink{Lemma}{lemma:#1}}%

\providecommand{\eqlab}[1]{}%
\renewcommand{\eqlab}[1]{\label{equation:#1}}
\newcommand{\Eqref}[1]{\HLinkSuffix{Eq.~(}{equation:#1}{)}}

%%%%%%%%%%%%%%%%%%%%%%%%%%%%%%%%%%%%%%%%%%%%%%%%%%%%%%%%%%%%%%%%%%% 
% Sariel's standard commands...
%%%%%%%%%%%%%%%%%%%%%%%%%%%%%%%%%%%%%%%%%%%%%%%%%%%%%%%%%%%%%%%%%%% 

\newcommand{\remove}[1]{}%
\newcommand{\Set}[2]{\left\{ #1 \;\middle\vert\; #2 \right\}}
\newcommand{\pth}[2][\!]{\mleft({#2}\mright)}%
\newcommand{\pbrcx}[1]{\left[ {#1} \right]}%
\newcommand{\Prob}[1]{\mathop{\mathbf{Pr}}\!\pbrcx{#1}}
\newcommand{\Ex}[2][\!]{\mathop{\mathbf{E}}#1\pbrcx{#2}}

\newcommand{\ceil}[1]{\left\lceil {#1} \right\rceil}
\newcommand{\floor}[1]{\left\lfloor {#1} \right\rfloor}

\newcommand{\brc}[1]{\left\{ {#1} \right\}}
\newcommand{\cardin}[1]{\left| {#1} \right|}%

\renewcommand{\th}{th\xspace}
\newcommand{\ds}{\displaystyle}%

\renewcommand{\Re}{\mathbb{R}}%
\newcommand{\reals}{\Re}%


%%%%%%%%%%%%%%%%%%%%%%%%%%%%%%%%%%%%%%%%%%%%%%%%%%%%%%%%%%%%%%%%%%%%%%%%%
% Defining comptenum environment using enumitem
\usepackage[inline]{enumitem}

\newlist{compactenumA}{enumerate}{5}%
\setlist[compactenumA]{topsep=0pt,itemsep=-1ex,partopsep=1ex,parsep=1ex,%
   label=(\Alph*)}%

\newlist{compactenuma}{enumerate}{5}%
\setlist[compactenuma]{topsep=0pt,itemsep=-1ex,partopsep=1ex,parsep=1ex,%
   label=(\alph*)}%

\newlist{compactenumI}{enumerate}{5}%
\setlist[compactenumI]{topsep=0pt,itemsep=-1ex,partopsep=1ex,parsep=1ex,%
   label=(\Roman*)}%

\newlist{compactenumi}{enumerate}{5}%
\setlist[compactenumi]{topsep=0pt,itemsep=-1ex,partopsep=1ex,parsep=1ex,%
   label=(\roman*)}%

\newlist{compactitem}{itemize}{5}%
\setlist[compactitem]{label=\ensuremath{\bullet}}%
\setlist[compactitem]{topsep=0pt,itemsep=-1ex,partopsep=1ex,parsep=1ex,%
   label=\ensuremath{\bullet}}%


\usepackage{stmaryrd}%
\providecommand{\IntRange}[1]{\mleft\llbracket #1 \mright\rrbracket}
\newcommand{\IRX}[1]{\IntRange{#1}}%
\newcommand{\IRY}[2]{\left\llbracket #1:#2 \right\rrbracket}

%%%%%%%%%%%%%%%%%%%%%%%%%%%%%%%%%%%%%%%%%%%%%%%%%%%%%%%%%%%%%%%%%%%%%%%%%%

\usepackage{wasysym}

\newcommand{\disk}{\Mh{\ocircle}}
\newcommand{\diskVY}[2]{\disk_{\downarrow}^{#1}\pth{#2}}%
\newcommand{\si}[1]{#1}



%%%%%%%%%%%%%%%%%%%%%%%%%%%%%%%%%%%%%%%%%%%%%%%%%%%%%%%%%%%%%%%%%%% 
%%%%%%%%%%%%%%%%%%%%%%%%%%%%%%%%%%%%%%%%%%%%%%%%%%%%%%%%%%%%%%%%%%% 
% Papers specific commands...
%%%%%%%%%%%%%%%%%%%%%%%%%%%%%%%%%%%%%%%%%%%%%%%%%%%%%%%%
%%%%%%%%%%%%%%%%%%%%%%%%%%%%%%%%%%%%%%%%%%%%%%%%%%%%%%%%

\providecommand{\Mh}[1]{#1}%

\newcommand{\eps}{\varepsilon}

\newcommand{\FF}{\Mh{\mathcal{F}}}%
\newcommand{\LL}{\mathcal{L}}
\newcommand{\DT}{\Mh{\mathcal{D}}}%
\newcommand{\DTX}[1]{\Mh{\mathcal{DT}}\pth{#1}}
\newcommand{\DG}{\Mh{\mathcal{D}}}%

\newcommand{\etal}{\textit{et~al.}\xspace}

\newcommand{\Term}[1]{\textsf{#1}}
\newcommand{\TermI}[1]{\Term{#1}\index{#1@\Term{#1}}}

\newcommand{\QSPD}{\Term{QSPD}\xspace}

\newcommand{\StavThanks}[1]{%
   \thanks{Department of Computer Science;
      University of Illinois; 201 N. Goodwin Avenue; Urbana, IL,
      61801, USA; {\tt stava2\atgen{}illinois.edu}; {\tt
         \url{https://publish.illinois.edu/stav-ashur}.} #1}}

\newcommand{\pa}{\Mh{p}}%
\newcommand{\pb}{\Mh{q}}%
\newcommand{\pc}{\Mh{u}}%
\newcommand{\pd}{\Mh{v}}%


\newcommand{\px}{\Mh{x}}%
\newcommand{\py}{\Mh{y}}%
\newcommand{\pz}{\Mh{z}}%

\newcommand{\dGY}[2]{\Mh{\mathsf{d}}\pth{#1,#2}}%
\newcommand{\dGZ}[3]{\Mh{\mathsf{d}_{#1}}\pth{#2,#3}}%
\newcommand{\dY}[2]{\left\| #1  #2 \right\|}%
\newcommand{\ddY}[2]{\left\| #1 - #2 \right\|}%

\newcommand{\angleX}[1]{\sphericalangle #1}


\newcommand{\dsZ}[3]{\Mh{\mathsf{d}}_{#1}\pth{#2, #3}}%
\newcommand{\dZ}[3]{\left\| #2 - #3 \right\|_{#1}}%

\newcommand{\dsY}[2]{\mathsf{d}\pth{#1,#2}}
\newcommand{\DistSetY}[2]{\dsY{#1}{#2}}%

\newcommand{\body}{\Mh{C}}%

\newcommand{\grid}{\Mh{\mathsf{K}}}%

\newcommand{\EucG}{\Mh{\EuScript{K}}_\PS}


\providecommand{\G}{\Mh{G}}%
\renewcommand{\G}{\Mh{G}}%
\newcommand{\GA}{\Mh{H}}%
%\newcommand{\GA}{\Mh{H}}%
\newcommand{\GB}{\Mh{I}}%

\providecommand{\GB}{\Mh{I}}%
\renewcommand{\GB}{\Mh{I}}%


\newcommand{\PS}{\Mh{P}}%
\newcommand{\PSA}{\Mh{Q}}%

\newcommand{\PX}{\Mh{X}}%
\newcommand{\PY}{\Mh{Y}}%

\newcommand{\DotProd}[2]{\permut{{#1},{#2}}}
\newcommand{\permut}[1]{\left\langle {#1} \right\rangle}

\newcommand{\Line}{\Mh{\ell}}%

\newcommand{\PSup}{\Mh{P}_\uparrow}%
\newcommand{\PSdown}{\Mh{P}_\downarrow}%

\newcommand{\QS}{\Mh{\mathcal{Q}}}%
\newcommand{\liftX}[1]{\mathrm{lift}\pth{#1}}%

\newcommand{\rect}{{\Mh{R}}}%

\newcommand{\EG}{\Mh{E}}%
\newcommand{\EGX}[1]{\Mh{E}\pth{#1}}%
\newcommand{\region}{\Mh{\mathcalb{r}}}%
\newcommand{\gminus}{-}%
\newcommand{\interiorX}[1]{\mathrm{int}\pth{#1}}%
\newcommand{\restrictY}[2]{#1 \cap {#2}}

\newcommand{\cpX}[1]{\Mh{\mathrm{c{}p}}\pth{#1}}%
\newcommand{\diamX}[1]{\mathrm{diam}\pth{#1}}%

\newcommand{\spread}{\Mh{\Phi}}
\newcommand{\spreadX}[1]{\spread\pth{#1}}

\newcommand{\WS}{\Mh{\mathcal{W}}}%
\newcommand{\WeightX}[1]{\Mh{\omega} \pth{#1}}
\newcommand{\diameterX}[1]{\mathrm{d{}i{}am}\pth{#1}}

\newcommand{\SSPD}{\Term{SSPD}\xspace}%

\newcommand{\PSB}{\Mh{B}}%
\newcommand{\PSC}{\Mh{C}}%

\newcommand{\PSX}{\Mh{X}}%
\newcommand{\PSY}{\Mh{Y}}%

\newcommand{\WSPD}{\Term{WSPD}\xspace}%

\newcommand{\coneY}[2]{\mathrm{cone}\pth{#1,#2}}%
\newcommand{\IS}{\Mh{\mathcal{I}}}%
\newcommand{\epsA}{\Mh{\vartheta}}%

\newcommand{\GY}[2]{\Mh{\mathcal{S}}\pth{#1, #2}}%
\newcommand{\cen}{\Mh{c}}%
\newcommand{\Pair}{\Mh{\Xi}}%

%\newcommand{\polylog}{\mathop{\mathrm{polylog}}}%
\newcommand{\XSays}[2]{{ {$\rule[-0.12cm]{0.2in}{0.5cm}$\fbox{\tt #1:}
      } #2 \marginpar{\textcolor{red}{#1}}
      {$\rule[0.1cm]{0.3in}{0.1cm}$\fbox{\tt
            end}$\rule[0.1cm]{0.3in}{0.1cm}$} } }
\newcommand{\sariel}[1]{{\XSays{Sariel}{#1}}}
\newcommand{\stav}[1]{{\XSays{Stav}{#1}}}

\newcommand{\QSup}{\QS_{\uparrow}}
\newcommand{\QSdown}{\QS_{\downarrow}}


%%%%%%%%%%%%%%%%%%%%%%%%%%%%%%%%%%%%%%%%%%%%%%%%%%%%%%%%%%%%%%%%%%
%%%%%%%%%%%%%%%%%%%%%%%%%%%%%%%%%%%%%%%%%%%%%%%%%%%%%%%%%%%%%%%%%%
%%%%%%%%%%%%%%%%%%%%%%%%%%%%%%%%%%%%%%%%%%%%%%%%%%%%%%%%%%%%%%%%%%
% Restating lemmas/theorems...
%
% Example
%---------------------------------------------------------------------
% \SaveContent{\LemmaNumVerticesDepthBody}{ BLA BLA }
% 
% \begin{lemma}[{{\normalfont Proof in \apndref{num:v:depth}}}]
%       \lemlab{num:vertices:depth}%
%       \LemmaNumVerticesDepthBody{}
% \end{lemma}
%...
% \bigskip%
% \RestatementOf{\lemref{num:vertices:depth}}{\LemmaNumVerticesDepthBody}
%---------------------------------------------------------------------

\newcommand{\SaveContent}[2]{%
   \expandafter\newcommand{#1}{#2}%
}

\newcommand{\RestatementOf}[2]{
   \noindent%
   \textbf{Restatement of #1.}
   % 
   {\em #2{}}%
}

%%% End
%%%%%%%%%%%%%%%%%%%%%%%%%%%%%%%%%%%%%%%%%%%%%%%%%%%%%%%%%%%%%%%%%%
%%%%%%%%%%%%%%%%%%%%%%%%%%%%%%%%%%%%%%%%%%%%%%%%%%%%%%%%%%%%%%%%%%


%%%%%%%%%%%%%%%%%%%%%%%%%%%%%%%%%%%%%%%%%%%%%%%%%%%%%%%%%%%%%%%%%%%%%%%%
%%%%%%%%%%%%%%%%%%%%%%%%%%%%%%%%%%%%%%%%%%%%%%%%%%%%%%%%%%%%%%%%%%%%%%%%
% \mathcalb - a different font that looks a bit like mathcal

\DeclareFontFamily{U}{BOONDOX-calo}{\skewchar\font=45 }
\DeclareFontShape{U}{BOONDOX-calo}{m}{n}{<-> s*[1.05] BOONDOX-r-calo}{}
\DeclareFontShape{U}{BOONDOX-calo}{b}{n}{<-> s*[1.05] BOONDOX-b-calo}{}
\DeclareMathAlphabet{\mathcalb}{U}{BOONDOX-calo}{m}{n}
\SetMathAlphabet{\mathcalb}{bold}{U}{BOONDOX-calo}{b}{n}
\DeclareMathAlphabet{\mathbcalb}{U}{BOONDOX-calo}{b}{n}

% \mathcalb - end of file
%%%%%%%%%%%%%%%%%%%%%%%%%%%%%%%%%%%%%%%%%%%%%%%%%%%%%%%%%%%%%%%%
%%%%%%%%%%%%%%%%%%%%%%%%%%%%%%%%%%%%%%%%%%%%%%%%%%%%%%%%%%%%%%%%

\newcommand{\CHX}[1]{\mathsf{ch}\pth{#1}}%

\newcommand{\rinX}[1]{\Mh{r}_{\mathrm{in}}\pth{#1}}%
\newcommand{\routX}[1]{\Mh{R}_{\mathrm{out}}\pth{#1}}%
\newcommand{\arX}[1]{\Mh{\mathsf{a{}r}}\pth{#1}}%
\newcommand{\Elp}{\Mh{\mathcal{E}}}

\newcommand{\cell}{\Mh{\mathsf{C}}}%

\newcommand{\Of}{\Mh{\mathcal{O}}}%
\newcommand{\Oeps}{\Mh{\mathcal{O}_\eps}}%
\newcommand{\gConst}{\Mh{\tau}}%
\newcommand{\xSlabX}[1]{{\protect\overleftrightarrow{#1}}}
\newcommand{\ySlabX}[1]{\updownarrow\!{#1}}
\newcommand{\widthX}[1]{\Mh{\mathsf{wd}}\pth{#1}} \smallskip%


\newcommand{\HERE}{%
   {\noindent\hspace{-1cm}\rule{1.2\linewidth}{4cm}}   
   % \rule{4cm}{4cm}
}

\SoCGVer{%
   \newcommand{\myparagraph}[1]{%
      \noindent%
      \textbf{#1.}
   }%
}
\NotSoCGVer{%
   \newcommand{\myparagraph}[1]{%
      \paragraph{#1}
   }%
}

\providecommand{\TPDF}[2]{\texorpdfstring{#1}{#2}}

%%%%%%%%%%%%%%%%%%%%%%%%%%%%%%%%%%%%%%%%%%%%%%%%%%%%%%%%
%%BeginIpePreamble
%%%%%%%%%%%%%%%%%%%%%%%%%%%%%%%%%%%%%%%%%%%%%%%%%%%%%%%%

\newcommand{\sqr}{\mathcalb{s}}%
\newcommand{\sqrA}{\mathcalb{t}}%
\newcommand{\sqrB}{\mathcalb{u}}%
\newcommand{\seg}{s}%
\newcommand{\origin}{o}%
\newcommand{\polylog}{\mathop{\mathrm{polylog}}}%

\newcommand{\Binfty}{\Mh{\mathcalb{b}_\infty}}%

\newcommand{\constA}{c_1}
\newcommand{\constB}{c_2}
\newcommand{\constC}{c_3}
\newcommand{\constD}{c_4}
\newcommand{\constE}{c_5}
\newcommand{\constF}{c_6}

%\newcommand{\interiorX}{\mathrm{int}\pth{#1}}
\newcommand{\diamC}{\Mh{\mathcalb{d}}}%

\newcommand{\cone}{\Mh{\mathcalb{c}}}%
\newcommand{\coneB}{c}%
\newcommand{\ConeSet}{\Mh{\EuScript{C}}}%
\newcommand{\dir}{\Mh{\mathsf{n}}}
\newcommand{\nnZ}[3]{\Mh{\mathsf{nn}}_{#1} \pth{#2, #3}}

\newcommand{\normX}[1]{\left\| #1 \right\|}

\newcommand{\Trap}{\Mh{T}}%
\newcommand{\Traps}{\Mh{\mathcal{T}}}%

\newcommand{\CC}{\Mh{\mathcal{C}}}%
\newcommand{\Body}{\CC}

\newcommand{\senseX}[1]{\Mh{\mathrm{sen}} \pth{#1}}%
\newcommand{\Triangles}{\bm\Delta}%


%%%%%%%%%%%%%%%%%%%%%%%%%%%%%%%%%%%%%%%%%%%%%%%%%%%%%%%%
%%EndIpePreamble
%%%%%%%%%%%%%%%%%%%%%%%%%%%%%%%%%%%%%%%%%%%%%%%%%%%%%%%%



